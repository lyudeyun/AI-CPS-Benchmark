%%% Local Variables:
%%% mode: latex
%%% TeX-master: t
%%% End:
%twocolumn,notitlepage,letterpaper,
\documentclass[10pt,a4paper]{article}   % Specifies the document style.
%\usepackage[latin1]{inputenc}
%\usepackage{Vmargin,afterpage,xspace,psfrag,rotating} 
%\usepackage{lastpage,fancyhdr,color,graphicx}           % 
\usepackage{amsmath}%,natbib,multicol}            % 
\usepackage{a4wide,xspace}
\textheight=24cm
\addtolength{\topmargin}{-20mm}
\addtolength{\textwidth}{10mm}
\addtolength{\oddsidemargin}{-10mm}
\addtolength{\evensidemargin}{-10mm}
\begin{document}    
\newcommand{\eg}{e.g.\xspace}
\newcommand{\ie}{i.e.,\xspace}

\newcommand{\etc}{etc.\@\xspace}
\renewcommand{\textfraction}{0.05}
\renewcommand{\topfraction}{0.95}
\renewcommand{\bottomfraction}{0.95}
\renewcommand{\floatpagefraction}{0.85}
\setcounter{totalnumber}{5}
%\renewcommand{\thefootnote}{\arabic{footnote}}
%\setmarginsrb{leftmargin}{topmargin}{rightmargin}{bottommargin}%
%                     {headheight}{headsep}{footheight}{footskip}
%\setpapersize{a4}
%\setmarginsrb{12mm}{17mm}{11mm}{13mm}{0pt}{0mm}{11mm}{12mm}
%\setmarginsrb{12mm}{17mm}{11mm}{13mm}{0pt}{0mm}{11mm}{30mm}
%\fancypagestyle{plain}{
%\fancyhf{}
%\fancyfoot[R]{ Page \thepage\ \hspace{0.3cm} of \hspace{0.3cm} \pageref{LastPage} \hspace{0in}}
%\renewcommand{\headrulewidth}{0pt}
%\renewcommand{\footrulewidth}{0pt}
%}
%\pagestyle{plain}
%\lhead{} \chead{} \rhead{}
%\lfoot{\hspace{1in} }
%\cfoot

\title{ WAFO Conventions and Architecture}
\author{ Per A. Brodtkorb}

\date{\today}
\maketitle
% double spacing between lines
%\renewcommand{\baselinestretch}{2}
%\small
%\normalsize

\section{Introduction}
This document describes how the Wave Analysis for Fatigue and Oceanography (WAFO) Toolbox
is organized and the conventions/rules for writing new functions
and demonstrations. 
It is intended to contributers and developers of WAFO to unify 
the source code and make maintenance of the toolbox easier.
\section{Architecture}
\begin{description}
\item[/WAFO] is the main directory containing different directories for the  WAFO software, datasets and documentation
\item[/WAFO/docs] contains the documentation for the toolbox both in ascii and postscript format.
\item[/WAFO/paper] is a subdirectory including scripts for reproducing figures in various articles and technical reports.
\item[/WAFO/demos] contains different demonstrations
\item[/WAFO/data] contains datasets used in the demo scripts.
\item[/WAFO/source/] contains mex and fortran source files
\item[/WAFO/exec/alpha] contains mex/fortran compiled executables for alpha computers
\item[/WAFO/exec/pc] contains mex/fortran compiled executables for PC
\item[/WAFO/....] (other directories of WAFO specified by Jesper?) 
\end{description}

\section{Documentation}
The documentation directory \verb+/WAFO/docs/+ contains all the documentation available for the WAFO. 
The contents of any of these files
may be examined by typing its name for ascii files or viewing in ghostview for postscriptfiles.
\begin{description}
\item[datastructures.m] Datastructures in WAFO 
\item[definitions.m] Definitions 
\item[install.m] How to install WAFO
\item[bugreport.m] How to report BUGS of the WAFO software
\item[gettingstarted.m] How and where to start with WAFO
\item[limitations.m] Known limitations of WAFO
\item[references.m] A list of all the articles books referenced in functions where the user can find more documentation
\item[tutorial.ps]  The tutorial of WAFO
\item[funcreference.ps] refence card of all the functions available in WAFO
\end{description}
Every directory  have a \verb+Contents.m+ file which list all the files in the 
directory along with a short description of what they do.   The first two lines of the \verb+Contents.m+
contains a directory description and a version marker of the form 
\begin{verbatim}
% Wave Analysis for Fatigue and Oceanography (WAFO) Toolbox.
% Version x.y.z  dd-mm-yyyy
\end{verbatim}
in order to get the toolbox to work with the MATLAB function \verb+ver+.
and to make maintenance of WAFO easier.
( Look at the Contents.m in any MathWorks toolbox (but
 not MATLAB) for an example).

Also each function must have a help header explaining their usage and purpose (see: help header).
The directory also contains some utility functions for documentation:
\begin{description}
\item[WAFOhelpfiles] all help files arranged by function name
\item[WAFOsynopsis] one line synopsis arranged by function name
\item[WAFOcontents] list all \verb+contents.m+ files
\item[WAFOfiles] list all WAFO files by directory
\item[mkcontents] automatically generate contents  for current directory
  based on H1 line of the files.
\item[mkref] automatically generate a reference file  for current directory
\end{description}


\subsection{Functions }
\subsubsection{General conventions}
Each line of code in the functions must not exceed 76 columns
to ease the readability of the code. Function names must be written in lower case letters
without any underscores. Also local variable names should 
preferably be in lower case letters. Global variables should be written in 
upper case letters in order to avoid confusion with local variables. 

\subsubsection{Help header}
Generally a function in WAFO should be well commented and include the following
blocks of information, each block separated by a blank line, in the given order in the help header:

\begin{description}
\item[Header:] (H1 line) Name of the function (in uppercase letters) followed by a
  concise and definitive description appropriate for retrieval purposes by
  the MATLAB \verb+lookfor+ command. Not to exceed one line.  \\
  Format: \\
\% $\langle$ FUNCTION NAME $\rangle $
  $\langle$ blank $\rangle $ $\langle$ Description $\rangle$
\item[Call:] Specifying the calling prototype, \ie output argument(s),
  function name and input argument(s). Optionally, more than one
  prototype call. Function name in lower case letters. \\
  Format:\\
 \%$\langle$ blank $\rangle $ CALL: $\langle$ 2 blanks $\rangle$
  $\langle$ output $\rangle $ $\langle$ blank $\rangle $ = $\langle$ blank $\rangle$
  $\langle$ function name $\rangle$ ($\langle$ input $\rangle$) \\
 \%$\langle$ 8 blanks $\rangle$ 
  $\langle$ output $\rangle $ $\langle$ blank $\rangle $ = $\langle$ blank $\rangle$
  $\langle$ function name $\rangle$ ($\langle$ input $\rangle$)
\item[Outputs/inputs:] A list describing the output and input arguments in
  the order they appear in the calling prototype. The description should
  also indicate the data type of the argument. An optional input argument is
  indicated by specifying the default value enclosed in parenthesis. Each
  line should be aligned so that all the equal-signs are in the same column.\\
  Format: \\
  \%  $\langle$ argument $\rangle$
  $\langle$ blank $\rangle$ = $\langle$ blank $\rangle$ $\langle$ description
  $\rangle$ (Default $\langle$ value $\rangle$ )
\item[Description:] Detailed explanation of what the function does, the
  assumptions made, the limitations and how the algorithm works.  This
  should be as detailed as possible without exceeding one page.
\item[Side effects:] If any like changing/using global variables, 
changing properties of figure windows etc.
should also be notified in the help header
\item[Example(s):] An example of how to use the function in practice is
  desirable.\\
  Format: \\
\% $\langle$ blank $\rangle$ Example: $\langle$ blank $\rangle$ $\langle$ Description $\rangle$ \\
\% $\langle$ 3 blanks $\rangle$ $\langle$ Command line(s) $\rangle$  \\
\item[See also:] Comma separated list of related functions or functions
  which this routine calls.
\end{description}

\subsubsection{Secret help}
Immediately after  the help header the following information should be given: 
\begin{description}
\item[References:] A complete reference from which the user can obtain further information
if it exist should also be included\\
  Format: \\
\% References \\
\% \\
\% $\langle$ Author(s) $\rangle$ ($\langle$ year $\rangle$ ) \\
\% $\langle$ Title $\rangle$  \\
\% $\langle$ Journal/Volume/Pages/Publisher/... $\rangle$
\item[Tested on:] Specifies on which matlab version the function has been
  tested.
\item[History:] Revision log of the function in chronological order. \\
  Format: \\
 \% revised by $\langle$ name $\rangle$ $\langle$ date $\rangle$  \\
 \% $\langle$ short description of the revision $\rangle$
\end{description}

An example on how it looks like in the \verb+torsethaugen+ function is given
on the next page.
\newpage{}
\begin{verbatim}
%TORSETHAUGEN Calculates and plots a doubly peaked spectral density 
%
% CALL:  [S Ss Sw] = torsethaugen(w,data); 
%        [S Ss Sw] = torsethaugen(w,data,plotflag); 
%
%   S, Ss, Sw =  a struct (See WAFOdefinitions) containing the spectral density 
%                for total, swell and wind, respectively.  
%        w    = angular frequency (default linspace(0,3,257))
%        data = [Hm0 Tp]
%               Hm0 = Significant wave height
%               Tp  = 2*pi/wp, primary peak period
%    plotflag = 0, do not plot the spectrum.
%               1, plot the  spectrum (default).
%
% The spectrum is written as
%         S(w)=Ss(w)+Sp(w) 
% where Ss and Sp are modified JONSWAP spectrums for the secondary 
% and primary peak, respectively. The energy
% is divided  between the two peaks according to empirical parameters
% which has been fitted to the average measured wave spectra from the North Sea.
% The dataset used consisted of 20 000 spectra divided into 146 different classes 
% of Hm0 and Tp ranging from 0.5 to 11 meters and  3.5 to 19 sec,
% respectively. See Torsethaugen (1996).
%
% Preliminary comparisons with spectra from other areas indicate that 
% some of the empirical parameters are dependent on geographical location.
% Thus the model must be used with care for other areas than the North Sea 
% and sea states outside the range of Hm0 and Tp as given above.
%
% Example : 
%   w=linspace(0,4,129);
%   [S Ss Sw] = torsethaugen(w,[7 12],0); 
%
% See also: jonswap, wspecplot

% References 
% 
%  Torsethaugen et. al. (1996)
%  Model for a doubly peaked wave spectrum 
%  Report No. STF22 A96204. 
%  SINTEF Civil and Environmental Engineering, Trondheim

% Tested on: Matlab 6.0
% 
% History:
% Revised by pab 27.06.1999
%    updated some documentation
% By pab 14.05.1999
\end{verbatim}
\section{Paper}
The \verb+/paper+ directory contains subdirectories including scripts for recreating figures in 
published articles. Each article have their own subdirectory.  The directories contains 
demonstration scripts to 
generate individual figures and (possibly) specialized tools/functions not available in the official 
release of WAFO for generating the figures. These are the rules for the scripts 
\begin{enumerate}
\item One script creates one complete figure.
\item If some variables set in one script is needed by another script they must be saved as global variables.
\item No \verb+pause+ or \verb+print+ statements allowed in scripts.
\end{enumerate}

The steps involved to add a demo are as follows :
\begin{enumerate}
\item Decide on a name for the demo, \eg{}, \verb+my+, and create a new subdirectory accordingly after this name
\item Create the following files:
\begin{description}
\item[mydemo.m] Starts a user controlled demonstration with choices. 
\item[myinit.m] set up data structures,globals
\item[myfig1,2...m] called from choices which creates one figure each.
 Each \verb+.m+-file for an individual figure contains a help 
header which should be displayed in the command window at execution time of the script
along with a short explanation to the figure.\\
 Format: \\
\% MYFIG $\langle$ Number $\rangle $:  $\langle$ Caption $\rangle $ \\
\% \\
\% $\langle$ short explanation $\rangle$
\item[myintro.m]help file describing the purpose of the demo
\item[mycleanup.m]clears all globals created by the demo
\end{description}
\item Specialized tools/functions not available in the official 
  release of WAFO needed for generating the figures should be put into 
  \verb+\private\+ subdirectory.
\end{enumerate}

\section{Demos}
Just  like the paper directory, 
the demos directory also contains different subdirectories with scripts producing figures. 
The only difference
is that it does not reproduce figures from published articles but merely test and 
demonstrate various methodologies,
and release code that approximately reproduces figures in other articles.  

\section{Datasets}
This directory contains 1D, 2D (3D) datasets used by the scripts in demo and paper directory.
%Also some utility functions for loading the dataset into a MATLAB session is included.
\subsection{Dataset format}
%\begin{enumerate}
%\item[1D:] 
The data is stored in vertical format with sampled values aligned  in
columns. The default suffix for the data files is \verb+.dat+
The documentation \verb+.m+-file explains what each column is a measure of.
Sampling times if given should be in the first column.

%\item[2D:] currently not available.
%\end{enumerate}


\subsection{Dataset documentation}
Each dataset in the system has a documentation \verb+.m+-file with the following structure:
\begin{description}
\item[Title:] name of the file in upper case letter followed by descriptive text. 
Not to exceed one line. \\
 Format: \\
\%  $\langle$ DATASET NAME $\rangle $
  $\langle$ blank $\rangle $ $\langle$ Description $\rangle $
\item[Call:] indicating how to access the data.\\
  Format:\\
 \%$\langle$ blank $\rangle $ CALL: $\langle$ 2 blanks $\rangle$
  $\langle$ output $\rangle $ = $\langle$ blank $\rangle $ = $\langle$ blank $\rangle$
  $\langle$ function name $\rangle$ ($\langle$ input $\rangle$)
\item[Size:] The size of the data.\\
Format:\\
\%$\langle$ blank $\rangle $ Size $\langle$  blanks $\rangle$
  : $\langle$ blanks $\rangle $   $\langle$ Size  $\rangle$
\item[Sampling rate:] 
Format:\\
\%$\langle$ blank $\rangle $ Sampling rate $\langle$  blanks $\rangle$
  : $\langle$ blanks $\rangle $   $\langle$ rate in Hz  $\rangle$
\item[Device:] Measuring device used to sample the data. \\
Format:\\
\%$\langle$ blank $\rangle $ Device $\langle$  blanks $\rangle$
  : $\langle$ blanks $\rangle $   $\langle$ name of measuring device $\rangle$
\item[Source:] Indication of the original source of the data.\\
Format:\\
\%$\langle$ blank $\rangle $ Source $\langle$  blanks $\rangle$
  : $\langle$ blanks $\rangle $   $\langle$ Source  $\rangle$
\item[Format:] Specifying the format of the file.\\
Format:\\
\%$\langle$ blank $\rangle $ Format $\langle$  blanks $\rangle$
  : $\langle$ blanks $\rangle $   $\langle$ Format  $\rangle$
 \item[Description:] and other relevant facts of the data , \ie{} where and when the 
data were measured. An indication of the quality of the data should also be included. 
Descriptive measures of the data like significant 
wave height, $H_{m0}$ , and peak period, $T_{p}$ may also be given here.  
Also restrictions on the use of the data 
if any should be given here. \\
Format:\\
\%$\langle$ blank $\rangle $ Description $\langle$  blanks $\rangle$
  :  $\langle$ Newline $\rangle $ \\
\%  $\langle$ Description  $\rangle$
\item[See also:] Comma separated list of related files of functions 
\end{description}
Example is given on the next page.
\newpage{}
\begin{verbatim}
%GULLFAKS  surface elevation measured at Gullfaks C 24.12.1990
%              
% CALL:  xn = load('gullfaks.dat')
%  
% Size           :    2560 X 2
% Sampling Rate  :    2.5 Hz
% Device         :    EMI laser
% Source         :    STATOIL 
% Format         :    ascii, c1: time c2: surface elevation
%
% Description    :
%  The wave data was measured 24th December 1990 at the Gullfaks C platform  
%  in the North Sea from 17.00 to  21.20. The  water depth of  218 m is regarded 
%  as deep water  for the most  important wave components.
%  There are two EMI laser sensors named $219$ and $220$. This data  set  is obtained from
%  sensor $219$, which is  located in the
%  Northwest corner  approximately two platform leg diameters away from   
%  the closest  leg. 
%  Thus the wave elevation is not expected  to be significantly 
%  affected by  diffraction effects for incoming waves in the western
%  sector.   The wind direction for this period is from the Southwest. 
%  Some  difficulties in calibration of the instruments have been reported
%  resulting in several consecutive measured values being equal or nearly equal 
%  in the observed data set.
% 
%  Hm0 = 6.8m, Tm02 = 8s, Tp = 10.5
%
% See also: gullfaks.jpg
\end{verbatim}
\end{document}    
